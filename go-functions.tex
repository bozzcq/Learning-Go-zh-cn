\epi{``I'm always delighted by the light touch and stillness of
early programming languages.  Not much text; a lot gets
done. Old programs read like quiet conversations
between a well-spoken research worker and a well-
studied mechanical colleague, not as a debate with a
compiler.  Who'd have guessed sophistication bought
such noise?''}{\textsc{RICHARD P. GABRIEL}}

\noindent{}Functions are the basic building blocks of Go programs; all interesting
stuff happens in them. A function is declared as follows:
\begin{lstlisting}[caption=A function declaration,label=src:function definition]
|\begin{tikzpicture}[overlay]
\ubrace{0.8,-1.5}{0.0,-1.5}{The keyword \key{func} is used to declare a function;}
%
\ubrace{2.9,-1.5}{0.9,-1.5}{A function can optionally be bound to a specific type. %
This is called the \first{\emph{receiver}}{receiver}. A function with a receiver is %
a \index{method}{method}. This will be explored in chapter \ref{chap:interfaces};}
%
\ubrace{4.6,-1.5}{3.1,-1.5}{\emph{funcname} is the name of your function;}
%
\ubrace{5.9,-1.5}{4.7,-1.5}{The variable \var{q} of type \type{int} is %
the input parameter. The parameters are passed %
\first{\emph{pass-by-value}}{pass-by-value} meaning they are copied;}
%
\ubrace{7.7,-1.5}{6.1,-1.5}{%
The variables \var{r} and \var{s} are the %
\index{named return parameters}{named return parameters} for this function. %
Functions in Go can have multiple return values, see section %
"\titleref{sec:multiple return}" on page \pageref{sec:multiple return}. %
If you want the return %
parameters not to be named you only give the types: %
\lstinline{(int,int)}. If you have only one value to return you may omit %
the parentheses. If your function is a subroutine and does not have %
anything to return you may omit this entirely;}
%
\ubrace{10.7,-1.5}{8.0,-1.5}{This is the function's body. Note that %
\func{return} is a statement so the braces around the parameter(s) are %
optional.}
\end{tikzpicture}|
type mytype int	|\coderemark{New type, see chapter \ref{chap:beyond}}|

func (p mytype) funcname(q int) (r,s int) { return 0,0 }
||
\end{lstlisting}

\showremarks
Here are two examples. The first is a function without a return value,
while the bottom one is a simple function that returns its input.

\begin{lstlisting}
func subroutine(in int) { return }
\end{lstlisting}

\begin{lstlisting}
func identity(in int) int { return in }
\end{lstlisting}

Functions can be declared in any order you wish. The compiler scans the
entire file before execution, so function prototyping is a thing of the
past in Go. Go disallows nested functions, but you can work around this with
anonymous functions. See section ``\titleref{sec:functions as values}'' on page
\pageref{sec:functions as values} in this chapter.

Recursive functions work just as in other languages:
\begin{lstlisting}[caption=Recursive function]
func rec(i int) {
   if i == 10 {
        return
   }
   rec(i+1)
   fmt.Printf("%d ", i)
}
\end{lstlisting}
This prints: \texttt{9 8 7 6 5 4 3 2 1 0}.
%%\newpage %% TODO don't want a newpage here actually
\section{Scope}
Variables declared outside any functions are \first{global}{scope!local} in Go, those
defined in functions are \first{local}{scope!local} to those functions. If names overlap --- a
local variable is declared with the same name as a global one --- the
local variable hides the global one when the current function is
executed.

\begin{minipage}{.5\textwidth}
\begin{lstlisting}[caption=Local scope,label=src:scope1]
|\begin{tikzpicture}[overlay]
\draw [->,thick] (3.1,-5.00) arc (-60:90:2.00cm);
\draw [->,thick] (3.1,-7.00) arc (-60:90:0.20cm);
\end{tikzpicture}|
package main

var a = 6

func main() {
    p()
    q()
    p()
}

func p() {
    println(a)
}

func q() {
    a := 5    |\coderemark{Definition}|
    println(a)
}
\end{lstlisting}

\hfill
\vfill
\end{minipage}
\hfill
\begin{minipage}{.5\textwidth}
\input{fig/scope2.tex}
\hfill
\vfill
\end{minipage}

In listing \ref{src:scope1} we introduce a local variable \var{a}
in the function \func{q()}.
The local \var{a} is only visible in \func{q()}. This is
why the code will print: \texttt{656}.
In listing \ref{src:scope2} no new variables are introduced, there
is only a global \var{a}.
Assigning a new value to \var{a} will be globally visible. This code will
print: \texttt{655}

In the following example we call \func{g()} from \func{f()}:

\lstinputlisting[caption=Scope when calling functions from functions]{src/scope3.go}

The output will be: \texttt{565}. A \emph{local} variable is \emph{only}
valid when we are executing the function in which it is defined. 
%%Finally, one can create a \first{"function literal"}{function literal} in which you essentially 
%%define a function inside another
%%function, i.e. a \first{nested function}{nested function}. 
%%The following figure should clarify why it prints: \texttt{565757}. 
%%\input{fig/scope3.tex}

\section{Multiple return values}
\label{sec:multiple return}
One of Go's unusual (for compiled languages) features is that functions and methods can return multiple
values (Python and Perl can do this too). This can be used to improve on a couple of 
clumsy idioms in C programs:
in-band error returns (such as -1 for \texttt{EOF}) and modifying an argument.
In Go, \lstinline{Write} returns a count and an
error: ``Yes, you wrote some bytes but not all of them because you filled the
device''. The signature of \lstinline{*File.Write} in package
\package{os} is:
\begin{lstlisting}
func (file *File) Write(b []byte) (n int, err error)
\end{lstlisting}
and as the documentation says, it returns the number of bytes written and a
non-\lstinline{nil} \var{error} when \lstinline{n != len(b)}. This is a common
style in Go.

In the absence of tuples as a native type, multiple return values are the next
best thing. You can return precisely what you want without
overloading the domain space with special values to signal errors.

\section{Named result parameters}
\label{sec:named result parameters}
The return or result parameters of a Go function can be given names and used
as regular variables, just like the incoming parameters. When named, they are
initialized to the zero values for their types when the function begins. If the
function executes a \key{return} statement with no arguments, the current values of
the result parameters are returned. Using these
features enables you (again) to do more with less code \footnote{This is
a motto of Go; ``Do \emph{more} with \emph{less} code''}.

The names are not mandatory but they can make code shorter and clearer:
\emph{they are documentation}. 
If we name the results of \lstinline{nextInt} it becomes obvious which
returned \type{int} is which.

\begin{lstlisting}
func nextInt(b []byte, pos int) (value, nextPos int) { /* ... */ }
\end{lstlisting}
Because named results are initialized and tied to an unadorned
\key{return},
they can simplify as well as clarify. Here's a version of
\lstinline{io.ReadFull} that uses them well:

\begin{lstlisting}
func ReadFull(r Reader, buf []byte) (n int, err error) {
    for len(buf) > 0 && err == nil {
        var nr int
        nr, err = r.Read(buf)
        n += nr
        buf = buf[nr:len(buf)]
    }
    return
}
\end{lstlisting}

\section{Deferred code}
\label{sec:deferred code}
Suppose you have a function in which you open a file and perform various
writes and reads on it. In such a function there are often spots where
you want to return early. If you do that, you will need to close the file
descriptor you are working on. This often leads to the following code:
\begin{lstlisting}[caption=Without defer]
func ReadWrite() bool {
    file.Open("file")
    // Do your thing
    if failureX {
	file.Close() |\coderemark{}|
	return false
    }

    if failureY {
	file.Close() |\coderemark{}|
	return false
    }
    file.Close() |\coderemark{}|
    return true
}
\end{lstlisting}
A lot of code is repeated here. To overcome this Go has the
\first{\key{defer}}{keyword!defer} statement. After
\key{defer} you specify a function which is called just \emph{before}
the current function exits.

The code above could be rewritten as follows. This makes the 
function more readable, shorter and puts the \func{Close} right next 
to the \func{Open}.
\begin{lstlisting}[caption=With defer]
func ReadWrite() bool {
    file.Open("file")
    defer file.Close()	|\coderemark{\func{file.Close()} is added to defer list}|
    // Do your thing
    if failureX {
	return false    |\coderemark{\func{Close()} is now done automatically}|
    }
    if failureY {
	return false    |\coderemark{And here too}|
    }
    return true         |\coderemark{And here}|
}
\end{lstlisting}
You can put multiple functions on the ``deferred list''\index{deferred list}, like this
example from \cite{effective_go}:
\begin{lstlisting}
for i := 0; i < 5; i++ { 
    defer fmt.Printf("%d ", i) 
} 
\end{lstlisting}
Deferred functions are executed in LIFO order, so the above code
prints: \lstinline{4 3 2 1 0}. 

With \func{defer} you can even change return values, provided that
you are using named result parameters and a function
literal\index{function!literal}\footnote{A function literal
is sometimes called a \index{closure} closure.}, i.e:
\begin{lstlisting}[caption=Function literal]
defer func() {
	/* ... */
}()		 |\coderemark{() is needed here}|
\end{lstlisting}
Or this example which makes it easier to understand why and where
you need the braces:
\begin{lstlisting}[caption=Function literal with parameters]
defer func(x int) {
	/* ... */
}(5)  |\coderemark{Give the input variable \var{x} the value 5}|
\end{lstlisting}
In this (unnamed) function you can access any named return
parameter:
\begin{lstlisting}[caption=Access return values within defer]
func f() (ret int) {|\coderemark{\var{ret} is initialized with zero}|
	defer func() {
		ret++|\coderemark{Increment \var{ret} with 1}|
	}()
	return 0|\coderemark{1 \emph{not} 0 will be returned!}|
}
\end{lstlisting}

\section{Variadic parameters}
Functions that take a variable number of parameters are known as variadic functions.
To declare a function as variadic:
\begin{lstlisting}
func myfunc(arg ...int) {}
\end{lstlisting}
The \lstinline{arg ...int} instructs Go to see this as a function that
takes a variable number of arguments. Note that these arguments all
have the type \type{int}. Inside your function's body the variable
\var{arg} is a slice of ints:
\begin{lstlisting}
for _, n := range arg {
    fmt.Printf("And the number is: %d\n", n)
}
\end{lstlisting}
If you don't specify the type of the variadic argument it defaults to the
empty interface \var{interface\{\}} (see chapter
\ref{chap:interfaces}).
Suppose we have another variadic function called \func{myfunc2}, the 
following example shows how to pass variadic arguments to it:
\begin{lstlisting}
func myfunc(arg ...int) {
    myfunc2(arg...)  |\coderemark{Pass it as-is}|
    myfunc2(arg[:2]...)  |\coderemark{Slice it}|
}
\end{lstlisting}

\section{Functions as values}
\label{sec:functions as values}
\index{function!as values}
\index{function!literals}
As with almost everything in Go, functions are also \emph{just} values.
They can be assigned to variables as follows:
\lstinputlisting[label=src:anonfunc,caption=Anonymous function,linerange={3,}]{src/anon-func.go}
If we use \lstinline{fmt.Printf("%T\n", a)} to print the type of
\var{a}, it prints \func{func()}.

Functions--as--values may be used in other places, for example maps.
Here we convert from integers to functions:
\begin{lstlisting}[caption=Functions as values in maps]
var xs = map[int]func() int{
    1: func() int { return 10 },
    2: func() int { return 20 },
    3: func() int { return 30 }, |\coderemark{Mandatory ,}|
    /* ... */
}
\end{lstlisting}
Or you can write a function that takes a function as its parameter, for
example a \func{Map} function that works on \type{int} slices. This is
left as an exercise for the reader (see exercise Q\ref{ex:map function}
on page \pageref{ex:map function}).

\section{Callbacks}
\label{sec:callbacks}
Because functions are values they are easy to pass to functions, from where
they can be used as callbacks. First define a function that
does ``something'' with an integer value:
\begin{lstlisting}
func printit(x int) {|\coderemark{Function returns nothing}|
    fmt.Printf("%v\n", x)|\coderemark{Just print it}|
}
\end{lstlisting}
The signature of this function is: \lstinline{func printit(int)}, or
without the function name: \mbox{\lstinline{func(int)}}. To create a new function
that uses this one as a callback we need to use this signature:
\begin{lstlisting}
func callback(y int, f func(int)) {|\coderemark{\func{f} has the function}|
    f(y)    |\coderemark{Call the callback \func{f} with \var{y}}|
}
\end{lstlisting}

\section{Panic and recovering}
\label{sec:panic}
Go does not have an exception mechanism, like that in Java for instance: you cannot throw exceptions.
Instead it uses a panic-and-recover mechanism. It is worth remembering that you should use this as
a last resort, your code will not look, or be, better if it is littered with panics. It's a powerful tool:
use it wisely. So, how do you use it?

The following description was taken from \cite{go_blog_panic}:
\begin{description}
\item[Panic]{is a built-in function that stops the ordinary flow of control and begins panicking. When the function 
\func{F} calls \key{panic},
execution of \func{F} stops, any deferred functions in \func{F} are executed normally, and 
then \func{F} returns to its caller. To the caller, \func{F} then
behaves like a call to \key{panic}. The process continues up the stack until all functions in the current 
goroutine have returned, at which point the program crashes. 

Panics can be initiated by invoking \func{panic} directly. They can also be caused by \emph{runtime errors}, such
as out-of-bounds array accesses.}

\item[Recover]{is a built-in function that regains control of a panicking goroutine. Recover is \emph{only} useful inside 
\emph{deferred} functions.

During normal execution, a call to \func{recover} will return \type{nil} and have no other effect. 
If the current goroutine is panicking, a call
to \func{recover} will capture the value given to \func{panic} and resume normal execution.}
\end{description}

This function checks if the function it gets as argument will panic when it is
executed\footnote{Copied from a presentation of Eleanor McHugh.}:
\begin{lstlisting}
func throwsPanic(f func()) (b bool) { |\longremark{We define a new function \func{throwsPanic} that takes %
a function as an argument (see ``\titleref{sec:functions as values}''). It returns true if \func{f} panics %
when run, else false;}|
    defer func() { |\longremark{We define a \func{defer} function that utilizes \func{recover}. If the %
current goroutine panics, this defer function will notice that. If \func{recover()} returns non-\var{nil} we set \var{b} %
to true;}|
        if x := recover(); x != nil {
            b = true
        }
    }()
    f() |\longremark{Execute the function we received as the argument;}|
    return |\longremark{Return the value of \var{b}. Because \var{b} is a named return parameter (page %
\pageref{sec:named result parameters}), we don't specify \var{b}.}|
}
\end{lstlisting}
\showremarks


\section{Exercises}
\input{ex-functions/ex-average.tex}

\input{ex-functions/ex-order.tex}

\input{ex-functions/ex-scope.tex}

\input{ex-functions/ex-stack.tex}

\input{ex-functions/ex-vararg.tex}

\input{ex-functions/ex-fib.tex}

\input{ex-functions/ex-map.tex}

\input{ex-functions/ex-minmax.tex}

\input{ex-functions/ex-bubblesort.tex}

\input{ex-functions/ex-funcfunc.tex}

\cleardoublepage
\section{Answers}
\shipoutAnswer
